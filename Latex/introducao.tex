\documentclass[a4,12pt]{book}

\usepackage{template_introducao}

% Parallelepiped definitions
\usetikzlibrary{quotes,arrows.meta}

\begin{document}

\chapter{Introdução}
\thispagestyle{empty}
Frações é certamente um dos tópicos que mais desafia o ensino e a aprendizagem da na matemática da educação básica. Justamente por isso, tanto se publicou sobre o assunto nas últimas décadas (para citar apenas algumas das mais utilizadas:  {\it Rational Number Project, Institute of Education Science} (2010), Van de Walle (2009) e Wu (2011)). Este texto, organizado como uma proposta didática,  reúne as reflexões e as discussões dos autores sobre o tema, amparadas por essas publicações e pela análise de livros didáticos de diversos países. A proposta aqui apresentada foi planejada para:

\begin{enumerate}[(i)]
\item  ser aplicada diretamente em sala de aula, como material didático destinado aos anos intermediários do ensino fundamental (do 4º ao 7º ano) e 
\item amparar a formação e o desenvolvimento profissional do professor que ensina matemática na educação básica.
\end{enumerate}

O texto concentra-se na abordagem inicial de frações como objeto matemático, buscando explorar o assunto a partir de atividades que visam à construção conceitual do tema e a conduzir os alunos a desenvolverem o raciocínio matemático amparados por reflexão e por discussão. Assim, as atividades visam a desafiar os alunos e a levá-los a estabelecer suas próprias conclusões sobre os assuntos tratados. As atividades valorizam a capacidade cognitiva dos alunos respeitando uma organização crescente de dificuldade. Espera-se com isso mudar a perspectiva do binômio quantidade/qualidade. No lugar de uma quantidade enorme de exercícios, são propostas poucas  atividades que exigem maior reflexão e aprofundamento dos conceitos. Assim, são evitadas atividades de simples observação e repetição de modelos e os tradicionais ``exercícios de fixação", que, pontuais, são apenas com o objetivo de desenvolver a fluência em procedimentos espeíficos (Por exemplo, os que envolvem a equivalência entre frações) . 

Uma característica particular deste material é o diálogo com o professor. No início de cada lição, há uma introdução dirigida especificamente ao professor que apresenta os objetivos da lição, uma discussão dos aspectos matemáticos que serão tratados, as dificuldades esperadas e algumas observações sobre os passos cognitivos envolvidos. Diferente dos livros didáticos tradicionais, em que, para o professor, há pequenas observações pontuais junto ao texto do aluno e um longo texto teórico anexo ao final do livro, nesta propostas a ``conversa" com o professor é permanente. Em cada atividade são realizadas discussões sobre os objetivos a serem alcançados, recomendações e sugestões metodológicas para sua execução e, quando pertinente, uma discussão sobre algum desdobramento do assunto tratado.

Entende-se que, nesta etapa da escolaridade, considerando o cotidiano próprio do aluno, o conceito de fração aparece ligado a  noções informais traduzidas por expressões como metade, terço, quartos, décimos e centésimos, por exemplo. Assim, nas primeiras duas lições, buscou-se utilizar a linguagem verbal e os conhecimentos anteriores dos estudantes sobre situações em que aquelas expressões são utilizadas para conduzir as primeiras abordagens, visando à introdução de um conhecimento mais organizado e formal sobre o assunto. Apenas posteriormente são introduzidas a linguagem e a simbologia próprias da matemáica. 

As lições 1 e 2 introduzem os conceitos elementares e a linguagem de frações a partir de situações concretas e de modelos contínuos. Na lição 1, as frações emergem de situações concretas amparadas pela linguagem verbal. Uma vez estabelecida a unidade, a expressão ``fração unitária" nomeia cada uma das partes da divisão da unidade em partes iguais. Nas atividades dessas lições a unidade está fortemente vinculada a um objeto concreto. Assim, por exemplo, a fração de uma torta, não é ainda tratada com a abstração própria do conceito de número, mas como uma fatia da torta em uma equipartição. Toma-se bastante cuidado com o papel da determinação da unidade e com a necessidade de uma ``equipartição" para a identificação de uma fração. A notação simbólica de frações e as frações não unitárias, incusive as maiores do que a unidade, surgem apenas na Lição 2. As frações com numerador diferente de 1 são apresentadas a partir da justaposição de frações unitárias com mesmo denominador ou simplesmente contando-se essas frações. Para isso, tem-se a representação pictórica como um apoio importante. Nessas lições, as atividades são quase majoritariamente para pintar, identificar, reconhecer, analisar e justificar. 

Na Lição 3, é exigida maior abstração dos alunos. Retoma-se a representação de números na reta numérica, enfatizando, no contexto das frações, a associação do segmento unitário à unidade. Os modelos visuais contínuos e a justaposição de partes correspondentes às frações unitárias são a base da proposta desenvolvida. A representação das frações na reta numérica é usada para amparar a abordagem da comparação de frações com um mesmo numerador e com um mesmo denominador. Além disso, são propostas atividades que tratam a comparação de frações a partir de uma referência.

A Lição 4 trata da equivalência de frações tendo como objetivo a sua função na comparação de duas frações quaisquer. O assunto é abordado utilizando-se representações equivalentes em modelos de área retangulares, em modelos de área circulares e na reta numérica. A inclusão de modelos diferentes é proposital pois, com isso, o aluno tem a oportunidade de perceber as mesmas propriedades em contextos diferentes,. Finalizando a lição, são propostas atividades que conduzem à exploração da propriedade das frações que garante que , dadas duas frações diferentes, é sempre possível determinar uma terceira fração que está entre elas (propriedade de densidade).

Somas e subtrações de frações são o tema da Lição 5. Amparado pela equivalência de frações, que permite determinar subdivisões comuns da unidade para expressar as frações envolvidas nos cálculos, a soma e a subtração de frações são tratadas a partir de problemas. Os significados e os contextos que caracterizam as operações de adição e de subtração são semelhantes àqueles que compõem a abordagem dessas operações com números naturais, promovendo assim uma continuidade conceitual dos tópicos estudados.

Este volume marca o início de um trabalho em desenvolvimento que será ampliado e complementado por novos volumes e novas edições. Para o volume 2 está prevista a complementação da abordagem das operações com frações, trazendo a multiplicação e a divisão envolvendo frações, a abordagem de frações em situações e modelos discretos e o uso de frações em contextos de razão e de proporção, além das porcentagens.

\end{document}
