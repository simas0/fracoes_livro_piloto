\documentclass{standalone}%\documentclass[a4paper,10pt]{scrartcl}

\usepackage[active,pdftex,tightpage]{preview}
\usepackage{tikz}
\usepackage[brazilian]{babel}

\PreviewEnvironment{tikzpicture}

\usetikzlibrary{calc}

%cor para fundo do Para o professor no html ffccff
\definecolor{professor}{RGB}{250,204,255}
%cor para fundo da resposta no html ddffff
\definecolor{resposta}{RGB}{221, 255, 255}

\begin{document}
%\maketitle
% 
% \pagestyle{empty}
%  \begin{tikzpicture}[scale=0.6]
%  \fill[professor] (-1,-3) rectangle (17,7);
%  \draw (0,0) rectangle (4,3);
%   \draw (0,3) rectangle (4,6);
%   \draw (0,1.5) -- (4,1.5);
%   \draw (0,3) -- (4,6);
%   \draw (5,3) node{$\Rightarrow$};
%   \draw (6,-0.25) rectangle (10,2.75);
%   \draw (6,3.25) rectangle (10,6.25);
%   \draw (8,-0.75) node{Metade dos ret\^angulos};
%   \draw (11,3) node{$\Rightarrow$};
%   \draw (12,-0.25) rectangle (16,2.75);
%   \draw (12,3.25) rectangle (16,6.25);
%   \draw (12,3.25) -- (16,6.25);
%   \draw (12,1.25) -- (16,1.25);
%   \draw (14,-0.75) node{Cada uma das 4};
%   \draw (14,-1.5) node{partes \'e metade};
%   \draw (14,-2.25) node{da metade ($\frac{1}{4}$)};
% \end{tikzpicture}


\begin{tikzpicture}[scale=1]
 \fill[resposta] (-1,-1) rectangle (5,7);
 \draw (0,0) rectangle (4,6);
 \draw (0.5,0) -- (0.5,6);
 \draw (1.5,0) -- (1.5,6);
 \draw (3.5,0) -- (3.5,6);
\end{tikzpicture}


\begin{tikzpicture}[scale=1]
 \fill[resposta] (-1,-1) rectangle (5,7);
 \draw (0,0) rectangle (4,6);
 \draw (0,0) -- (2,6);
 \draw (1,0) -- (3,6);
 \draw (3.5,0) -- (3.5,6);
\end{tikzpicture}


\end{document}
