\documentclass{book}
\usepackage[margin=.5in]{geometry}


\usepackage[brazil]{babel}
\usepackage{template_professor}

\usepackage{todonotes}

\usepackage{microtype}

\usepackage{mdframed}

\begin{document}
\begin{multicols}{2}

\todo{1}


\subsection{Atividade 1}


  \begin{imagem*}[breakable]{}{}
    PÁGINA DE REPRODUÇÃO

    - FIGURA ARTÍSTICA -

    Imagens: a mesma barra de chocolate que aparece na figura correspondente à atividade, dividida exatamente como na imagem (instrução a seguir), e as partes separadas para serem reproduzidas isoladamente (como no exemplo).

        \includegraphics[width=120pt, keepaspectratio]{../../livro/media/cap1/secoes/chocolate_2.jpg}

        \includegraphics[width=120pt, keepaspectratio]{../../livro/media/cap1/secoes/chocolate_3.jpg}
  \end{imagem*}


\subsection{Atividade 2}

  \begin{imagem*}[breakable]{}{}      - FIGURA ARTÍSTICA
    \begin{nota*}[breakable]{}{}       NA PÁGINA PARA REPRODUCAO - INCLUIR a imagem das 3 pizzas repartidas: inteiras e com as respectivas partes isoladas.
    \end{nota*}

        \includegraphics[width=\textwidth, keepaspectratio]{../../livro/media/cap1/secoes/licao1_atv2.png}
    ilustração: Cambrainha

  \end{imagem*}

\subsection{Atividade 5}

  \begin{imagem*}[breakable]{}{}
    \begin{nota*}[breakable]{}{}       - FIGURA GEOMÉTRICA - PÁGINA PARA REPRODUÇÃO
      os retângulos devem estar organizados em duas páginas conforme a ilustração.
\\
      \includegraphics[width=\textwidth, keepaspectratio]{../../livro/media/undefined/quartos_encarte_1.jpg}
\\
      \includegraphics[width=\textwidth, keepaspectratio]{../../livro/media/cap1/secoes/quartos_encarte_2.jpg}
    \end{nota*}
  \end{imagem*}

\subsection{Atividade 9}
  

  \begin{imagem*}[breakable]{}{}
    \begin{nota*}[breakable]{}{}       PÁGINA PARA REPRODUÇÃO - Na página para reprodução deve conter as figuras da imagem. Estas imagens devem ter as dimensões especificadas na figura do exercício.
    \end{nota*}
  \end{imagem*}



\end{multicols}

\end{document}